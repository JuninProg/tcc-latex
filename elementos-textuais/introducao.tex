% ----------------------------------------------------------
% Introdução (com numeração)
% ----------------------------------------------------------
\chapter{Introdução}
% ----------------------------------------------------------

A amamentação é a primeira e mais importante contribuição que toda mãe pode dar à saúde do seu filho, assim como à sua própria saúde. No entanto, apesar dos esforços contínuos da Organização Mundial de Saúde em fundamentar pilares para disseminação de informações e recomendações, ainda enfrentamos desafios significativos na promoção e apoio ao aleitamento materno.

Segundo a OMS \citeonline{oms9241561300}, as autoridades competentes devem implementar medidas sociais para dar suporte e proteção às gestantes e mães. Isso implica em manter as mulheres adequadamente informadas sobre assuntos relacionados à alimentação infantil, que recebam suporte apropriado da família e comunidade a fim de facilitar e encorajar a amamentação, inibindo quaisquer influências contrárias.

Neste contexto, a partir da parceria entre o programa municipal de aleitamento materno da secretaria de saúde de Osório e o IFRS Campus Osório, foi criado o aplicativo Pró-Mamá e seu respectivo painel administrativo para controle dos servidores públicos.O \emph{app} foi lançado em agosto de 2018 e na época estava disponível para \emph{download} na \emph{play store} e \emph{app store}, qualquer usuário, mas principalmente as mães podiam acessar e obter informações sobre o aleitamento materno, registrar dúvidas e acompanhar os marcos do desenvolvimento de seu bebê.

Apesar do seu sucesso, de acordo com o \citeonline{promamapremiacao} o aplicativo conquistou o 1º lugar na categoria “Atenção Básica” no 31º Congresso do Conselho das Secretarias Municipais de Saúde do Rio Grande do Sul (Cosems/RS), atualmente encontra-se em estado defasado e em inconformidade com a Lei Geral de Proteção de Dados (LGPD), o que ocasionou na remoção do aplicativo nas lojas dos dispositivos móveis.

O presente trabalho tem como objetivo reestruturar o Painel Pró-Mamá para atender as necessidades das leis de privacidade de dados dos usuários e suas restrições de segurança, além de implantar novas tecnologias para facilitar o suporte de forma eficiente e segura no futuro. Dessa forma, não só contribuindo para manter as funcionalidades já estabelecidas, mas também agregar com novas soluções propostas pelos próprios servidores ao longo da observação do uso do painel desde seu lançamento.

\section{OBJETIVOS}

\subsection{Objetivo geral}

Reconstrução do painel e adição de novas funcionalidades para garantir conformidade com a LGPD e melhorar a experiência do usuário.

\subsection{Objetivos específicos}

\begin{itemize}
  \item Analisar os requisitos funcionais, não-funcionais e regras de negócio do Painel Pró-Mamá;
  \item Planejar arquitetura do painel administrativo e API;
  \item Estruturar banco de dados e operação para migração dos dados retroativos;
  \item Desenvolver painel e API;
  \item Manutenção e adição de novas funcionalidades para o sistema.
\end{itemize}

\section{JUSTIFICATIVA}

De acordo com \citeonline{onlineintervention} a meta-análise mostra que o uso de modelos de intervenção pela Internet podem melhorar o nível de conhecimento sobre o aleitamento materno e que o seu domínio está intimamente relacionado ao efeito da amamentação, sendo uma condição importante para promover o comportamento de amamentação e melhorar a taxa de aleitamento materno. Algumas puérperas, especialmente primíparas, carecem de experiência de amamentação, enquanto o modo de intervenção pela Internet facilita o aprendizado, através de plataformas ou aplicativos de educação sobre amamentação.

Neste contexto, o programa Pró-Mamá e suas tecnologias caracterizam-se por agentes desses modelos interventivos e desde sua implantação em 2019, seus resultados corroboram o estudo citado acima, são mais de 10.864 acessos, 1.348 cadastros realizados e mais de 188 dúvidas respondidas \citeonline{dadospromama}.

Portanto, se faz necessário que sejam feitas as devidas atualizações no Painel Pró-Mamá e seu ecossistema para adequar as medidas de segurança em conformidade com a LGPD e consequentemente desbloqueio nas respectivas lojas dos aplicativos móveis, restabelecendo assim um projeto que se faz tão presente no município de Osório e região.
