% ----------------------------------------------------------
% Introdução (com numeração)
% ----------------------------------------------------------
\chapter{Introdução}
% ----------------------------------------------------------

A amamentação é a primeira e mais importante contribuição que toda mãe pode dar à saúde do seu filho, assim como à sua própria saúde. No entanto, apesar dos esforços contínuos da Organização Mundial de Saúde
em fundamentar pilares para disseminação de informações e recomendações, ainda enfrentamos desafios significativos na promoção e apoio ao aleitamento materno.

Segundo os autores do artigo (9241561300.pdf (who.int)) (arrumar citação), as autoridades competentes devem implementar medidas sociais para dar suporte e proteção às gestantes e mães, sejam elas primíparas ou multíparas. Isso implica em manter as mulheres adequadamente informadas sobre assuntos relacionados à alimentação infantil, que recebam suporte apropriado da família e comunidade a fim de facilitar e encorajar a amamentação, inibindo quaisquer influências contrárias.

Neste contexto, a partir da parceria entre o programa municipal de aleitamento materno da secretaria de saúde de Osório e o IFRS Campus Osório, foi criado o aplicativo Pró-Mamá e seu respectivo painel administrativo para controle dos servidores públicos. O app foi lançado em agosto de 2018 e na época estava disponível para download na play store e app store, qualquer usuário, mas principalmente as mães podiam acessar e obter informações sobre o aleitamento materno, registrar dúvidas e acompanhar os marcos do desenvolvimento de seu bebê.

Apesar do seu sucesso, o aplicativo conquistou o 1º lugar na categoria “Atenção Básica” no 31º Congresso do Conselho das Secretarias Municipais de Saúde do Rio Grande do Sul (Cosems/RS), atualmente encontra-se em estado defasado e em inconformidade com a Lei Geral de Proteção de Dados (LGPD), o que ocasionou na remoção do aplicativo nas lojas dos dispositivos móveis. (Aplicativo desenvolvido pelo campus é premiado em congresso da área da saúde - Campus Osório (ifrs.edu.br)).

O presente trabalho tem como objetivo reestruturar o Painel Pró-Mamá para atender as necessidades das leis de privacidade de dados dos usuários e suas restrições de segurança, além de implantar novas tecnologias para facilitar o suporte de forma eficiente e segura no futuro. Dessa forma, não só contribuindo para manter as funcionalidades já estabelecidas, mas também agregar com novas soluções propostas pelos próprios servidores ao longo da observação do uso do painel desde seu lançamento.
