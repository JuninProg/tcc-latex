\chapter{Referencial Teórico}

\section{Node.js}

\citeonline{nodejs} é um \emph{software} de código aberto que foi criado em 2009 por Ryan Dahl, multiplataforma, baseado no interpretador V8 do Google e que permite a execução de códigos JavaScript fora de um navegador \emph{web}. A plataforma se destaca pelo seu modelo de E/S não bloqueante e orientado a eventos, o que torna ideal para aplicações em tempo real e de alta concorrência.

Além disso, o Node.js possui um ecossistema vasto e crescente, com milhares de pacotes disponíveis através de seu gerenciador de pacotes, o npm (\emph{Node Package Manager}). Isso facilita a integração de funcionalidades mais complexas e permite a criação de projetos de forma mais eficiente.

\section{React}

\citeonline{react} é uma biblioteca JavaScript de código aberto criada pelo Facebook em 2013. Ela permite o desenvolvimento de componentes reutilizáveis que podem ser combinados para a criação de \emph{interfaces} \emph{web} dinâmicas de forma prática.

O React facilita a produção de aplicações \emph{front-end} de alta perfomance e manuteníveis graças a separação de responsabilidades por componentes. Cada módulo possui seu estado e se renderiza a partir de suas parametrizações e interações com o usuário.

\section{Lei Geral de Proteção de Dados (LGPD)}

A Lei Geral de Proteção de Dados nº \citeonline{lgpd}, sancionada em agosto de 2018, estabelece diretrizes para a coleta, armazenamento, tratamento e compartilhamento de dados pessoais no Brasil.

A LGPD introduz princípios importantes que devem ser seguidos por todas as organizações que lidam com dados pessoais, o presente trabalho se enquadra também por meio de suas tecnologias, por isso é preciso entendê-la para aplicar suas diretrizes de maneira segura e transparente. Entre as obrigações impostas pela lei, destaca-se a necessidade de obter consentimento explícito dos titulares dos dados, a implementação de medidas de segurança adequadas e a notificação de incidentes de segurança.